\documentclass{article}

% Language setting
% Replace `English' with e.g. `Spanish' to change the document language
\usepackage[english]{babel}
\usepackage[fontset=ubuntu]{ctex}

% Set page size and margins
% Replace `letter paper' with`a4paper' for UK/EU standard size
\usepackage[a4paper,top=1.5cm,bottom=1.5cm,left=2cm,right=2cm,marginparwidth=1.75cm]{geometry}

% Useful packages
\usepackage{amsmath}
\usepackage{graphicx}
\usepackage{listings}
\usepackage{enumitem}
\usepackage[ruled]{algorithm2e}
\usepackage[colorlinks=true, allcolors=blue]{hyperref}

\SetKwProg{Function}{function}{:}{end}

\title{\heiti 计算机学院《算法设计与分析》第二次作业}
\author{114514 李田所}

\begin{document}
\maketitle

\section{数组填充问题}

李华有一个长度为$n$的整数数组$a$,这个数组有以下性质:

\begin{enumerate}[itemindent=3em]
    \item 这个数组的所有元素之和为$3$的倍数
    \item 这个数组的每个元素$a_i$都满足$a_i\in[l,r]$
\end{enumerate}

李华忘记了这个数组的元素,请你设计一个高效的算法,帮助他找出有多少个满足条件的数组,并分析该算法的时间复杂度。

例如,长度为$n=2$,满足区间$l=1,r=3$的数组,包括$[1,2],[2,1],[3,3]$,则答案为$3$。

\subsection{算法思想}

一个暴力的做法搜索全部可能的数组并逐个验证,有指数时间复杂度。

一个常规的做法是通过递推计算,记$dp[i][j]$为长度为$i$且总和模$3$余$j$的数组的个数,$count[i]$为区间$[l,r]$中和模$3$余$j$的数的个数,则有递推方程:

$$
dp[i][j]=\sum_{k=0}^2dp[i-1][k]\times{count}[(j-k+3)\%3];
$$

显然上述做法有线性时间复杂度。

一个更好的做法是使用矩阵快速幂加速上述递推,记$(a_i,b_i,c_i)$为长度为$i$且总和模$3$分别余$0,1,2$的数组的个数,则有状态转移矩阵:

$$
\begin{pmatrix}
    count[0]&count[1]&count[2]\\
    count[2]&count[0]&count[1]\\
    count[1]&count[2]&count[0]
\end{pmatrix}
$$

使下式成立:

$$
\begin{pmatrix}
    a_i&b_i&c_i
\end{pmatrix}
\cdot
\begin{pmatrix}
    count[0]&count[1]&count[2]\\
    count[2]&count[0]&count[1]\\
    count[1]&count[2]&count[0]
\end{pmatrix}
=
\begin{pmatrix}
    a_{i+1}&b_{i+1}&c_{i+1}
\end{pmatrix}
$$

由此可得:

$$
\begin{pmatrix}
    a_1&b_1&c_1
\end{pmatrix}
\cdot
\begin{pmatrix}
    count[0]&count[1]&count[2]\\
    count[2]&count[0]&count[1]\\
    count[1]&count[2]&count[0]
\end{pmatrix}
^{n-1}
=
\begin{pmatrix}
    a_n&b_n&c_n
\end{pmatrix}
$$

通过矩阵快速幂,可以在对数时间复杂度完成矩阵幂运算,且矩阵确定为三阶,其余部分均可以在$O(1)$时间内完成计算,故上述做法有对数时间复杂度。

上述主要做法均使用$C++$进行了实现并进行了测试,为使格式美观相关代码见文末附录。

\subsection{普通递推做法伪代码与复杂度分析}

\begin{algorithm}[H]

\caption{数组填充问题普通递推做法}
\LinesNumbered
\KwIn{整数$n,\ l,\ r$,为数组长度及左右端点}
\KwOut{整数$ans$,符合条件的数组个数}

\Function{$count\_from\_one(mod,num)$}{
    $ans\leftarrow{num}/3$\;
    \If{$num\%3=1\ and\ mod=1$}{
        $ans\leftarrow{ans}+1$\;
    }
    \If{$num\%3=2\ and\ mod\ge1$}{
        $ans\leftarrow{ans}+1$\;
    }
    \textbf{$return\ ans$}\;
}

\Function{$count\_between(mod,l,r)$}{
    $gap\leftarrow{max}(l,r,-l,-r)/3\cdot3+3$\;
    \textbf{$return\ count\_from\_one(mod,r+gap)-count\_from\_one(mod,l-1+gap)$}\;
}

\Function{$main(n,l,r)$}{
    \If{$l>r$}{
        \textbf{$return\ 0$}\;
    }
    \For{$i\leftarrow0\ to\ 2$}{
        $dp[0][i]\leftarrow{count\_between}(i,l,r)$\;
    }
    \For{$i\leftarrow1\ to\ n-1$}{
        \For{$j\leftarrow0\ to\ 2$}{
            \For{$k\leftarrow0\ to\ 2$}{
                $dp[i][j]\leftarrow{dp}[i][j]+dp[i-1][k]\cdot{dp}[0][(j-k+3)\%3]$\;
            }
        }
    }
    \textbf{$return\ dp[n-1][0]$}\;
}

\end{algorithm}

显然函数$count\_from\_one(mod,num)$和$count\_between(mod,l,r)$都是$O(1)$复杂度的,而主函数中,$19-21$行的循环只执行常数次,因此是$O(1)$复杂度的,$22-28$行的循环中只有最外层循环执行$n-1$次,其余均只执行常数次,因此是$O(n)$复杂度的,从而普通递推做法的总时间复杂度为$O(n)$。

\subsection{矩阵快速幂做法伪代码与复杂度分析}

\begin{algorithm}[H]

\caption{数组填充问题矩阵快速幂做法}
\LinesNumbered
\KwIn{整数$n,\ l,\ r$,为数组长度及左右端点}
\KwOut{整数$ans$,符合条件的数组个数}

\Function{$count\_from\_one(mod,num)$}{
    $ans\leftarrow{num}/3$\;
    \If{$num\%3=1\ and\ mod=1$}{
        $ans\leftarrow{ans}+1$\;
    }
    \If{$num\%3=2\ and\ mod\ge1$}{
        $ans\leftarrow{ans}+1$\;
    }
    \textbf{$return\ ans$}\;
}

\Function{$count\_between(mod,l,r)$}{
    $gap\leftarrow{max}(l,r,-l,-r)/3\cdot3+3$\;
    \textbf{$return\ count\_from\_one(mod,r+gap)-count\_from\_one(mod,l-1+gap)$}\;
}

\Function{$mult\_count(count,matrix)$}{
    \For{$i\leftarrow0\ to\ 2$}{
        $temp[i]\leftarrow{count}[i]$\;
        $count[i]\leftarrow0$\;
    }
    \For{$i\leftarrow0\ to\ 2$}{
        \For{$j\leftarrow0\ to\ 2$}{
            $count[j]\leftarrow{count}[j]+matrix[i][j]\cdot{temp}[i]$\;
        }
    }
}

\Function{$mult\_matrix(matrix)$}{
    \For{$i\leftarrow0\ to\ 2$}{
        \For{$j\leftarrow0\ to\ 2$}{
            $temp[i][j]\leftarrow{matrix}[i][j]$\;
            $matrix[i][j]\leftarrow0$\;
        }
    }
    \For{$i\leftarrow0\ to\ 2$}{
        \For{$j\leftarrow0\ to\ 2$}{
            \For{$k\leftarrow0\ to\ 2$}{
                $matrix[i][j]\leftarrow{matrix}[i][j]+temp[i][k]\cdot{temp}[k][j]$\;
            }
        }
    }
}

\end{algorithm}

\begin{algorithm}[H]

\caption{数组填充问题矩阵快速幂做法}
\LinesNumbered
\KwIn{整数$n,\ l,\ r$,为数组长度及左右端点}
\KwOut{整数$ans$,符合条件的数组个数}

\Function{$main(n,l,r)$}{
    \If{$l>r$}{
        \textbf{$return\ 0$}\;
    }
    \For{$i\leftarrow0\ to\ 2$}{
        $count[i]\leftarrow{count\_between}(i,l,r)$\;
    }
    $matrix[0][0]\leftarrow{count}[0],matrix[0][1]\leftarrow{count}[1],matrix[0][2]\leftarrow{count}[2]$\;
    $matrix[1][0]\leftarrow{count}[2],matrix[1][1]\leftarrow{count}[0],matrix[1][2]\leftarrow{count}[1]$\;
    $matrix[2][0]\leftarrow{count}[1],matrix[2][1]\leftarrow{count}[2],matrix[2][2]\leftarrow{count}[0]$\;
    $t\leftarrow{n}-1$\;
    \While{t>0}{
        \If{$t\%2=1$}{
            $mult\_count(count,matrix)$\;
        }
        $mult\_matrix(matrix)$\;
        $t=t/2$\;
    }
    \textbf{$return\ count[0]$}\;
}

\end{algorithm}

显然函数$count\_from\_one(mod,num)$、$count\_between(mod,l,r)$、$mult\_count(count,matrix)$和$mult\_matrix(matrix)$都是$O(1)$复杂度的,而主函数中,$5-7$行的循环只执行常数次,因此是$O(1)$复杂度的,$12-18$行的循环执行次数为$log_2(n-1)$,因此是$O(log n)$复杂度的,从而矩阵快速幂做法的总时间复杂度为$O(log n)$。

\section{最长递增子序列问题}

递增子序列是指:从原序列中按顺序挑选出某些元素组成一个新序列,并且该新序列中的任意一个元素均大于该元素之前的所有元素。例如,对于序列$<5,24,8,17,12,45>$,该序列的两个递增子序列为$<5,8,12,45>$和$<5,8,17,45>$,并且可以验证它们也是原序列最长的递增子序列。请设计算法来求出一个包含$n$个元素的序列$A=<a_1,a_2,···,a_n>$中的最长递增子序列,并分析该算法的时间复杂度。

\subsection{算法思想}

一个常规的做法是通过递推计算,一种递推的方式为记$dp[i]$为前$i$个数中以$a[i]$结尾的最长递增子序列长度,则有递推方程:

$$
dp[i]=max(dp[j])+1
$$

其中$j$需要满足:

$$
a[i]>a[j]
$$

则最终结果为:

$$
max(dp[1],dp[1],...dp[n])
$$

而在$dp[i]=max(dp[j])+1$成立时同时进行以下记录:

$$
last[i]=j
$$

则可从最终结果的下标开始向前寻找,最终得到一个最长递增子序列。

注意到枚举和下标$i$,对每个$i$寻找最大值和利用$last$寻找最长递增子序列都是线性时间复杂度的操作,上述做法有平方时间复杂度。

一个更好的做法是采用另一种递推的方式,记$dp[i]$为长度为$i$的递增子序列中最小的末尾元素的下标,此时可得递推方程:

$$
dp[pos]=i
$$

其中$pos$需要满足:

$$
a[dp[1]],a[dp[2]],...,a[dp[pos-1]]<a[i]
$$
$$
a[dp[pos]],a[dp[pos+1]],...,a[dp[n]]\ge{a}[i]
$$

则最终结果为:

$$
max(i)\ s.t.\ dp[i]\ne{null}
$$

而在$dp[pos]=i$成立时同时进行以下记录:

$$
last[i]=dp[pos-1]
$$

则可从最终结果的下标开始向前寻找,最终得到一个最长递增子序列。

注意到$dp$是单调不减的,可以使用二分查找获得$pos$,而对$i$的枚举和利用$last$寻找最长递增子序列都是线性时间复杂度的操作,故上述做法有线性乘对数的时间复杂度。

上述主要做法均使用$C++$进行了实现并进行了测试,为使格式美观相关代码见文末附录。

\subsection{普通递推做法伪代码与复杂度分析}

\begin{algorithm}[H]

\caption{最长递增子序列问题普通递推做法}
\LinesNumbered
\KwIn{整数$n$,为数组长度,长度为$n$的整数数组$a$}
\KwOut{整数$ans$,为最长递增子序列长度,长度为$ans$的整数数组$arr$,为最长递增子序列}

\Function{$main(n,m,a[])$}{
    \For{$i\leftarrow1\ to\ n$}{
        $dp[i]\leftarrow1$\;
    }
    \For{$i\leftarrow1\ to\ n$}{
        \For{$j\leftarrow1\ to\ i-1$}{
            \If{$a[i]>a[j]\ and\ dp[j]+1\ge{dp}[i]$}{
                $dp[i]\leftarrow{dp}[j]+1$\;
                $last[i]\leftarrow{j}$\;
            }
        }
    }
    $ans\leftarrow-1$\;
    $index\leftarrow-1$\;
    \For{$i\leftarrow1\ to\ n$}{
        \If{$dp[i]>ans$}{
            $ans\leftarrow{dp}[i]$\;
            $index\leftarrow{i}$\;
        }
    }
    \For{$i\leftarrow0\ to\ ans-1$}{
        $arr[ans-1-i]\leftarrow{a}[index]$\;
        $index\leftarrow{last}[index]$\;
    }
    \textbf{$return\ ans,arr$}\;
}

\end{algorithm}

主函数中,$2-4$行的循环执行次数为$n$,因此是$O(n)$复杂度的,$5-12$行的循环的执行次数为$\sum_{i=1}^n(i-1)$,因此是$O(n^2)$复杂度的,$15-20$行的循环执行次数为$n$,因此是$O(n)$复杂度的,$21-24$行的循环本质是在求一个最长递增子序列,最坏情况下会循环$n$次(输入序列单调递增时),因此是$O(n)$复杂度的,从而普通递推做法的总时间复杂度为$O(n^2)$。

\subsection{二分优化递推做法伪代码与复杂度分析}

\begin{algorithm}[H]

\caption{最长递增子序列问题二分优化递推做法}
\LinesNumbered
\KwIn{整数$n$,为数组长度,长度为$n$的整数数组$a$}
\KwOut{整数$ans$,为最长递增子序列长度,长度为$ans$的整数数组$arr$,为最长递增子序列}

\Function{$main(n,m,a[])$}{
    $ans\leftarrow-1$\;
    $index\leftarrow-1$\;
    $a[0]\leftarrow0x7fff\_ffff$\;
    \For{$i\leftarrow1\ to\ n$}{
        $l\leftarrow0$\;
        $r\leftarrow{n}+1$\;
        \While{$r-l>1$}{
            $mid\leftarrow(l+r)/2$\;
            \eIf{$a[dp[mid]]\ge{a}[i]$}{
                $r\leftarrow{mid}$\;
            } {
                $l\leftarrow{mid}$\;
            }
        }
        $dp[r]\leftarrow{i}$\;
        $last[i]\leftarrow{dp}[r-1]$\;
        \If{$r>ans$}{
            $ans\leftarrow{r}$\;
            $index\leftarrow{i}$\;
        }
    }
    \For{$i\leftarrow0\ to\ ans-1$}{
        $arr[ans-1-i]\leftarrow{a}[index]$\;
        $index\leftarrow{last}[index]$\;
    }
    \textbf{$return\ ans,arr$}\;
}

\end{algorithm}

主函数中,$5-22$行的循环的执行次数为$n$,而每次循环中,需要进行$8-15$行的二分查找,由于二分查找是$O(log n)$复杂度的,$6-21$行的循环是$O(n log n)$复杂度的,$23-26$行的循环本质是在求一个最长递增子序列,最坏情况下会循环$n$次(输入序列单调递增时),因此是$O(n)$复杂度的,从而二分优化递推做法的总时间复杂度为$O(n log n)$。

\section{硬币问题}

给定$n$枚硬币($n$为奇数),编号为$1,2,···,n$。投掷第$i$枚硬币时有$p_i$的概率正面朝上,有$1−p_i$的概率反面朝上。

设计算法求解投掷这$n$枚硬币,其中正面朝上的硬币数量多于反面朝上的硬币数量的概率,并分析该算法的时间复杂度。

例如给定$n=3$枚硬币,其正面朝上的概率分别为$p_1=0.3,p_2=0.6,p_3=0.8$。有下述四种情况正面朝上的硬币数量多于反面朝上:

\begin{enumerate}[itemindent=3em]
    \item 三枚硬币同时朝上,概率为$0.3\times0.6\times0.8=0.144$。
    \item 第一枚硬币朝下,第二枚硬币朝上,第三枚硬币朝上,概率为$0.7\times0.6\times0.8=0.336$。
    \item 第一枚硬币朝上,第二枚硬币朝下,第三枚硬币朝上,概率为$0.3\times0.4\times0.8=0.096$。
    \item 第一枚硬币朝上,第二枚硬币朝上,第三枚硬币朝下,概率为$0.3\times0.6\times0.2=0.036$。
\end{enumerate}

故总概率为$0.144+0.336+0.096+0.036=0.612$。

\subsection{算法思想}

一个暴力的想法是按照题目要求枚举$n$个硬币共计$2^n$种可能的朝向及对应的概率,并累加符合条件的结果,有指数时间复杂度。

一个常规的做法是通过递推计算,记$dp[i][j]$为前$i$个硬币中有$j$个朝上的概率,则有递推方程:

$$
dp[i][j]=dp[i-1][j-1]\times{p}[i]+dp[i-1][j]\times(1-p[i])
$$

上述做法会填满一个$n$行$n$列的表格,且每个单元格可以在$O(1)$时间内计算出结果,故上述做法有平方时间复杂度。

一个更好的做法是利用$\prod(p_ix+1-p_i)$使用分治$FFT$进行求解,可以达到$O(n\cdot{log}n\cdot{log}n)$的时间复杂度,考虑到上述做法过程较为繁琐,在此不做展开(逃)。

上述主要做法均使用$C++$进行了实现并进行了测试,为使格式美观相关代码见文末附录。

\subsection{递推做法伪代码与复杂度分析}

\begin{algorithm}[H]

\caption{硬币问题递推做法}
\LinesNumbered
\KwIn{整数$n$,为数组长度,长度为$n$的浮点数数组$a$,为每个硬币正面朝上的概率}
\KwOut{浮点数$ans$,为正面朝上的硬币数量多于反面朝上的硬币数量的概率}

\Function{$main(n,a[])$}{
    $dp[0][0] \leftarrow 1$\;
    \For{$i\leftarrow1\ to\ n$}{
        $dp[i][0]\leftarrow{dp}[i-1][0]\cdot(1-a[i])$\;
    }
    \For{$i\leftarrow1\ to\ n$}{
        \For{$j\leftarrow1\ to\ n$}{
            $dp[i][j]\leftarrow{dp}[i-1][j-1]\cdot{a}[i]+dp[i-1][j]\cdot(1-a[i])$\;
        }
    }
    $ans\leftarrow0$\;
    \For{$i\leftarrow{n/2+1}\ to\ n$}{
        $ans\leftarrow{ans}+dp[n][i]$\;
    }
    \textbf{$return\ ans$}\;
}

\end{algorithm}

主函数中,$3-5$行的循环执行次数为$n$,因此是$O(n)$复杂度的,$6-10$行的循环的执行次数为$n^2$,因此是$O(n^2)$复杂度的,$12-14$行的循环执行次数为$n/2-1$,因此是$O(n)$复杂度的,从而递推做法的总时间复杂度为$O(n^2)$。

\section{鲜花组合问题}

花店共有$n$种不同颜色的花,其中第$i$种库存有$a_i$枝,现要从中选出$m$枝花组成一束鲜花。请设计算法计算有多少种组合一束花的方案,并分析该算法时间复杂度。(每种花数量均相同的
方案算一种方案)

\subsection{算法思想}

一个暴力的想法是按照多重集组合数计算公式求解$2^n$个组合数得到结果,有指数时间复杂度。

一个常规的做法是通过递推计算,记$dp[i][j]$为从前$i$种花中选择$j$个的方案数,则有递推方程:

$$
dp[i][j]=\sum_{k=0}^{min(j,a[i])}dp[i-1][j-k];
$$

上述做法会填满一个$n$行$n$列的表格,且每个单元格可以在线性时间内计算出结果,故上述做法有三次方时间复杂度。

注意到上述做法中$\sum_{k=0}^{min(j,a[i])}dp[i-1][j-k]$的计算存在重复,即有如下两式:

$$
dp[i][j]=\sum_{k=0}^{min(j,a[i])}dp[i-1][j-k];
$$
$$
dp[i][j-1]=\sum_{k=0}^{min(j-1,a[i])}dp[i-1][j-k-1];
$$

考虑上面两式的关系,可以分为分下面三种情况:

$$
a[i]\ge{j}+1,\ dp[i][j]=dp[i][j-1]+dp[i-1][j]
$$
$$
a[i]=j,\ dp[i][j]=dp[i][j-1]+dp[i-1][j]
$$
$$
a[i]\le{j}-1,\ dp[i][j]=dp[i][j-1]+dp[i-1][j]-dp[i-1][j-1-a[i]]
$$

合并相同的情况,可以得到新的递推方程:

$$
a[i]\ge{j},\ dp[i][j]=dp[i][j-1]+dp[i-1][j]
$$
$$
a[i]<j,\ dp[i][j]=dp[i][j-1]+dp[i-1][j]-dp[i-1][j-1-a[i]]
$$

上述做法会填满一个$n$行$n$列的表格,且每个单元格可以在$O(1)$时间内计算出结果,故上述做法有平方时间复杂度。

一个更好的做法是利用生成函数使用分治$FFT$进行求解,可以达到$O(n\cdot{log}n\cdot{log}m)$的复杂度,考虑到上述做法过程较为繁琐,在此不做展开(逃)。

上述主要做法均使用$C++$进行了实现并进行了测试,为使格式美观相关代码见文末附录。

\subsection{普通递推做法伪代码与复杂度分析}

\begin{algorithm}[H]

\caption{鲜花组合问题普通递推做法}
\LinesNumbered
\KwIn{整数$n$,为花的种数,整数$m$,为选出个数,长度为$n$的整数数组$a$,为每种花的个数}
\KwOut{整数$ans$,为总方案数}

\Function{$main(n,m,a[])$}{
    \For{$i\leftarrow0\ to\ n$}{
        $dp[i][0]\leftarrow1$\;
    }
    \For{$i\leftarrow1\ to\ n$}{
        \For{$j\leftarrow1\ to\ m$}{
            \For{$k\leftarrow0\ to\ min(j,a[i])$}{
                $dp[i][j]\leftarrow{dp}[i][j]+dp[i-1][j-k]$\;
            }
        }
    }
    \textbf{$return\ dp[n][m]$}\;
}

\end{algorithm}

主函数中,$2-4$行的循环执行次数为$n$,因此是$O(n)$复杂度的,$5-11$行的循环中,最外层和中间层执行次数分别是$n,m$,最内层的执行次数为$min(j,a_i)\le{min}(m,max(a_1,a_2,...,a_n)$,因此$5-11$行的循环是$O(n\cdot{m}\cdot{min(m,max(a_1,a_2,...,a_n)})$复杂度的,从而普通递推做法的总时间复杂度为$O(n\cdot{m}\cdot{min(m,max(a_1,a_2,...,a_n)})$。

\subsection{优化递推做法伪代码与复杂度分析}

\begin{algorithm}[H]

\caption{鲜花组合问题优化递推做法}
\LinesNumbered
\KwIn{整数$n$,为花的种数,整数$m$,为选出个数,长度为$n$的整数数组$a$,为每种花的个数}
\KwOut{整数$ans$,为总方案数}

\Function{$main(n,m,a[])$}{
    \For{$i\leftarrow0\ to\ n$}{
        $dp[i][0]\leftarrow1$\;
    }
    \For{$i\leftarrow1\ to\ n$}{
        \For{$j\leftarrow1\ to\ m$}{
            \eIf{$a[i]\ge{j}$}{
                $dp[i][j]\leftarrow{dp}[i][j-1]+dp[i-1][j]$\;
            } {
                $dp[i][j]\leftarrow{dp}[i][j-1]+dp[i-1][j]-dp[i-1][j-1-a[i]]$\;
            }
        }
    }
    \textbf{$return\ dp[n][m]$}\;
}

\end{algorithm}

主函数中,$2-4$行的循环执行次数为$n$,因此是$O(n)$复杂度的,$5-13$行的循环中,外层和内层执行次数分别是$n,m$,每次循环可以在$O(1)$复杂度内完成计算,因此$5-13$行的循环是$O(n m)$复杂度的,从而优化递推做法的总时间复杂度为$O(n m)$。

\section{最大分值问题}

给定一个包含$n$个整数的序列$a_1,a_2,...,a_n$,对其中任意一段连续区间$a_i...a_j$,其分值为

$$
(\sum_{t=i}^j{a_t})\%p
$$

符号$\%$表示取余运算符,可以认为$p$远小于$n$。

现请你设计算法计算将其分为$k$段(每段至少包含$1$个元素)后分值和的最大值,并分析该算法的时间复杂度。

例如,将$3,4,7,2$分为$3$段,模数为$p=10$,则可将其分为$(3,4),(7),(2)$这三段,其分值和为$(3+4)\%10+7\%10+2\%10=16$。

\subsection{算法思想}

一个常规的做法是通过递推计算,记$dp[i][j]$为将前$i$个数分成$j$组时各组分值$(\sum{a_k})\%p$和的最大值,则有递推方程:

$$
dp[i][j]=max(score(i-k,i)+dp[i-k-1][j-1]),\ k=0,1,...,i-1
$$

上述做法会填满一个$n$行$min(n,m)$列的表格,且通过前缀和计算$score(i-k,i)$时,每个单元格可以在线性时间内计算出结果,故上述做法有三次方时间复杂度。

一个更好的做法是利用$p$远小于$n$优化,优化的关键在于以下性质:

$$
dp[i][j]+(\sum_{k=i+1}^n{a_k})\%p\le{dp}[i+t][j]+(\sum_{k=i+t+1}^n{a_k})\%p,\ t>0,\  dp[i][j]\equiv{dp}[i+t][j]\ (mod\ p)
$$

下面证明上述不等式的正确性,由于$dp[i][j]$表示将前$i$个数分成$j$组时各组分值$(\sum{a_t})\%p$和的最大值,其与前$i$个数和的差为$p$的整数倍,有:

$$
dp[i][j]=\sum_{k=1}^ia_k-\alpha{p},\ \alpha\in{N}
$$

记$dp[i][j]\equiv{dp}[i+t][j]\equiv{l}\ (mod\ p)$,则有:

$$
\sum_{k=1}^ia_k-\alpha_1p\equiv\sum_{k=1}^{i+t}a_k-\alpha_2p\equiv{l}\ (mod\ p)
$$
$$
\sum_{k=1}^ia_k\equiv\sum_{k=1}^{i+t}a_k\equiv{l}\ (mod\ p)
$$
$$
\sum_{k=1}^ia_k\equiv\sum_{k=1}^ia_k+\sum_{k=i+1}^{i+t}a_k\equiv{l}\ (mod\ p)
$$
$$
\sum_{k=i+1}^{i+t}a_k\equiv0\ (mod\ p)
$$
$$
\sum_{k=i+1}^{i+t}a_k=\alpha{p},\ \alpha\in{N}
$$

此时考虑$(\sum_{k=i+1}^n{a_k})\%p$和$(\sum_{k=i+t+1}^n{a_k})\%p$的情况:

$$
\sum_{k=i+1}^n{a_k}=\sum_{k=i+1}^{i+t}a_k+\sum_{k=i+t+1}^n{a_k}
$$
$$
(\sum_{k=i+1}^n{a_k})\%p=(\sum_{k=i+1}^{i+t}a_k+\sum_{k=i+t+1}^n{a_k})\%p
$$
$$
(\sum_{k=i+1}^n{a_k})\%p=(\alpha{p}+\sum_{k=i+t+1}^n{a_k})\%p
$$
$$
(\sum_{k=i+1}^n{a_k})\%p=(\sum_{k=i+t+1}^n{a_k})\%p
$$

另一方面,由于$dp[i][j]$表示将前$i$个数分成$j$组时各组分值$(\sum{a_k})\%p$和的最大值,而$dp[i+t][j]$表示将前$i+t$个数分成$j$组时各组分值$(\sum{a_k})\%p$和的最大值,记使$dp[i][j]$最大时,$j$个组分别为$group_{(i,1)},group_{(i,2)},...,group_{(i,j)}$,考虑将前$i+t$个数分成$j$组时的情况:

$$
group_{(i+t,1)}=group_{(i,1)}
$$
$$
group_{(i+t,2)}=group_{(i,2)}
$$
$$
...
$$
$$
group_{(i+t,j-1)}=group_{(i,j-1)}
$$
$$
group_{(i+t,j)}=group_{(i,j)}\cup(\bigcup_{k=i+1}^{i+t}a_k)
$$

显然这是一种可行的分组,且由于$\sum_{k=i+1}^{i+t}a_k\equiv0\ (mod\ p)$,此分解下各组分值$(\sum{a_k})\%p$和的最大值为$dp[i][j]$,又因为$dp[i+t][j]$表示将前$i+t$个数分成$j$组时各组分值$(\sum{a_k})\%p$和的最大值,可得:

$$
dp[i][j]\le{dp}[i+t][j],\ t>0,\  dp[i][j]\equiv{dp}[i+t][j]\ (mod\ p)
$$
$$
dp[i][j]+(\sum_{k=i+1}^n{a_k})\%p\le{dp}[i+t][j]+(\sum_{k=i+t+1}^n{a_k})\%p,\ t>0,\  dp[i][j]\equiv{dp}[i+t][j]\ (mod\ p)
$$

由于上述性质,我们只要在处理每个$i$时,向前枚举$p$个同余类中的$p$个末尾下标,即可将计算$dp[i][j]$的复杂度从$O(n)$调整到了$O(p)$,由于$p$远小于$n$,这种优化下的时间复杂度是更优的。

上述主要做法均使用$C++$进行了实现并进行了测试,为使格式美观相关代码见文末附录。

\subsection{普通递推做法伪代码与复杂度分析}

\begin{algorithm}[H]

\caption{最大分值问题普通递推做法}
\LinesNumbered
\KwIn{整数$n$,为数组长度,整数$m$,为分的段数,整数$p$,为模数,长度为$n$的整数数组$a$}
\KwOut{整数$ans$,为最大分值}

\Function{$main(n,m,p,a[])$}{
    \For{$i\leftarrow1\ to\ n$}{
        $sum[i]\leftarrow{a}[i]+sum[i-1]$\;
        $dp[i][0]\leftarrow0x8000\_0000$\;
    }
    \For{$i\leftarrow1\ to\ n$}{
        \For{$j\leftarrow1\ to\ min(m,i)$}{
            \For{$k\leftarrow0\ to\ i-1$}{
                $dp[i][j]\leftarrow{max}(dp[i][j],(sum[i]-sum[i-k-1])\%p+dp[i-k-1][j-1]$\;
            }
        }
    }
    \textbf{$return\ dp[n][m]$}\;
}

\end{algorithm}

主函数中,$2-5$行的循环执行次数为$n$,因此是$O(n)$复杂度的,$6-12$行的循环中,执行次数为:

$$
\sum_{i=1}^n\sum_{j=1}^{min(k,i)}i
$$

若$k<=n$,成立:

$$
\sum_{i=1}^n\sum_{j=1}^{min(k,i)}i\le\sum_{i=1}^n\sum_{j=1}^{k}i=\sum_{j=1}^{k}\sum_{i=1}^ni=k\sum_{i=1}^ni=O(k\cdot{n^2})=O(n^2\cdot{min}(n,k))
$$

否则,成立:

$$
\sum_{i=1}^n\sum_{j=1}^{min(k,i)}i=\sum_{i=1}^n\sum_{j=1}^ii=\sum_{i=1}^n i^2=O(n^3)=O(n^2\cdot{min}(n,k))
$$

因此$6-12$行的循环是$O(n^2\cdot{min}(n,k))$复杂度的,从而普通递推做法的总时间复杂度为$O(n^2\cdot{min}(n,k))$。

\subsection{优化递推做法伪代码与复杂度分析}

\begin{algorithm}[H]

\caption{最大分值问题优化递推做法}
\LinesNumbered
\KwIn{整数$n$,为数组长度,整数$m$,为分的段数,整数$p$,为模数,长度为$n$的整数数组$a$}
\KwOut{整数$ans$,为最大分值}

\Function{$main(n,m,p,a[])$}{
    \For{$i\leftarrow1\ to\ n$}{
        $sum[i]\leftarrow{a}[i]+sum[i-1]$\;
        $dp[i][0]\leftarrow0x8000\_0000$\;
    }
    \For{$i\leftarrow1\ to\ n$}{
        \For{$j\leftarrow1\ to\ min(m,i)$}{
            \For{$k\leftarrow0\ to\ p-1$}{
                $dp[i][j]\leftarrow{max}(dp[i][j],(sum[i]-sum[last[j-1][k]])\%p+dp[last[j-1][k]][j-1]$\;
            }
            $last[j][dp[i][j]\%p]\leftarrow{i}$;
        }
    }
    \textbf{$return\ dp[n][m]$}\;
}

\end{algorithm}

主函数中,$2-5$行的循环执行次数为$n$,因此是$O(n)$复杂度的,$6-13$行的循环中,最外层和最内层循环的执行次数分别是$n,p$,中间层循环的执行次数是$min(k,i)\le{min}(n,k)$,因此$6-13$行的循环是$O(n\cdot{min}(n,k)\cdot{p})$复杂度的,从而优化递推做法的总时间复杂度为$O(n\cdot{min}(n,k)\cdot{p})$。

\section{附录暨部分算法的实现}

\subsection{数组填充问题普通递推做法}

\definecolor{mygreen}{rgb}{0,0.6,0}
\definecolor{mygray}{rgb}{0.5,0.5,0.5}
\definecolor{mymauve}{rgb}{0.58,0,0.82}

\lstset{
    backgroundcolor=\color{white},   % choose the background color; you must add \usepackage{color} or \usepackage{xcolor}
    basicstyle=\footnotesize,        % the size of the fonts that are used for the code
    breakatwhitespace=false,         % sets if automatic breaks should only happen at whitespace
    breaklines=true,                 % sets automatic line breaking
    captionpos=bl,                   % sets the caption-position to bottom
    commentstyle=\color{mygreen},    % comment style
    deletekeywords={...},            % if you want to delete keywords from the given language
    escapeinside={\%*}{*)},          % if you want to add LaTeX within your code
    extendedchars=true,              % lets you use non-ASCII characters; for 8-bits encodings only, does not work with UTF-8
    frame=single,                    % adds a frame around the code
    keepspaces=true,                 % keeps spaces in text, useful for keeping indentation of code (possibly needs columns=flexible)
    keywordstyle=\color{blue},       % keyword style
    language=c++,                    % the language of the code
    morekeywords={*,...},            % if you want to add more keywords to the set
    numbers=left,                    % where to put the line-numbers; possible values are (none, left, right)
    numbersep=5pt,                   % how far the line-numbers are from the code
    numberstyle=\tiny\color{mygray}, % the style that is used for the line-numbers
    rulecolor=\color{black},         % if not set, the frame-color may be changed on line-breaks within not-black text (e.g. comments (green here))
    showspaces=false,                % show spaces everywhere adding particular underscores; it overrides 'showstringspaces'
    showstringspaces=false,          % underline spaces within strings only
    showtabs=false,                  % show tabs within strings adding particular underscores
    stepnumber=1,                    % the step between two line-numbers. If it's 1, each line will be numbered
    stringstyle=\color{mymauve},     % string literal style
    tabsize=2,                       % sets default tabsize to 2 spaces
}

\begin{lstlisting}
#include <cstdio>
#include <cstring>
#include <algorithm>
using namespace std;

const int maxn=10000;
int dp[maxn+5][3],n,l,r;

inline int cal(int mod,int num)
{
    int ans=num/3;
    if (num%3==1 && mod==1) ans++;
    if (num%3==2 && mod>=1) ans++;
    return ans;
}

inline int init(int mod)
{
    int gap=max(max(l,r),max(-l,-r))/3*3+3;
    return cal(mod,r+gap)-cal(mod,l-1+gap);
}

int main()
{
    while (~scanf("%d %d %d",&n,&l,&r))
    {
        if (l>r) {puts("0"); continue;}
        memset(dp,0,sizeof(dp));
        for (int i=0;i<3;i++) dp[0][i]=init(i);
        for (int i=1;i<n;i++)
            for (int j=0;j<3;j++)
                for (int k=0;k<3;k++)
                    dp[i][j]+=dp[i-1][k]*dp[0][(j-k+3)%3];
        printf("%d\n",dp[n-1][0]);
    }
    return 0;
}
\end{lstlisting}

\subsection{数组填充问题矩阵快速幂做法}

\begin{lstlisting}
#include <cstdio>
#include <algorithm>
using namespace std;

const int maxn=10000;
int matrix[3][3],cnt[3],n,l,r;

inline int cal(int mod,int num)
{
    int ans=num/3;
    if (num%3==1 && mod==1) ans++;
    if (num%3==2 && mod>=1) ans++;
    return ans;
}

inline int init(int mod)
{
    int gap=max(max(l,r),max(-l,-r))/3*3+3;
    return cal(mod,r+gap)-cal(mod,l-1+gap);
}

void mulCount()
{
    int tmp[3];
    for (int i=0;i<3;i++)
        tmp[i]=cnt[i],cnt[i]=0;
    for (int i=0;i<3;i++)
        for (int j=0;j<3;j++)
            cnt[j]+=matrix[i][j]*tmp[i];
    return;
}

void mulMatrix()
{
    int tmp[3][3];
    for (int i=0;i<3;i++)
        for (int j=0;j<3;j++)
        tmp[i][j]=matrix[i][j],matrix[i][j]=0;
    for (int i=0;i<3;i++)
        for (int j=0;j<3;j++)
            for (int k=0;k<3;k++)
                matrix[i][j]+=tmp[i][k]*tmp[k][j];
    return;
}

int main()
{
    while (~scanf("%d %d %d",&n,&l,&r))
    {
        if (l>r) {puts("0"); continue;}
        for (int i=0;i<3;i++) cnt[i]=init(i);
        matrix[0][0]=cnt[0],matrix[0][1]=cnt[1],matrix[0][2]=cnt[2];
        matrix[1][0]=cnt[2],matrix[1][1]=cnt[0],matrix[1][2]=cnt[1];
        matrix[2][0]=cnt[1],matrix[2][1]=cnt[2],matrix[2][2]=cnt[0];
        for (int t=n-1;t;t>>=1)
        {
            if (t&1) mulCount();
            mulMatrix();
        }
        printf("%d\n",cnt[0]);
    }
    return 0;
}
\end{lstlisting}

\subsection{最长递增子序列问题普通递推做法}

\begin{lstlisting}
#include <cstdio>

const int maxn=1000;
int n,a[maxn+5],dp[maxn+5],last[maxn+5];

int main()
{
    scanf("%d",&n);
    for (int i=1;i<=n;i++) scanf("%d",&a[i]),dp[i]=1;
    for (int i=1;i<=n;i++)
        for (int j=1;j<i;j++)
            if (a[i]>a[j] && dp[j]+1>=dp[i])
                dp[i]=dp[j]+1,last[i]=j;
    int ans=-1,idx=-1,tot=0;
    for (int i=1;i<=n;i++) if (dp[i]>ans) ans=dp[i],idx=i;
    while (idx) dp[tot++]=a[idx],idx=last[idx];
    printf("%d\n",ans);
    for (int i=tot-1;i>=0;i--) printf("%d ",dp[i]);
    puts(""); return 0;
}
\end{lstlisting}

\subsection{最长递增子序列问题二分优化递推做法}

\begin{lstlisting}
#include <cstdio>

const int maxn=1000;
int n,a[maxn+5],dp[maxn+5],last[maxn+5];

int main()
{
    scanf("%d",&n);
    int ans=-1,idx=-1,tot=0;
    for (int i=1;i<=n;i++) scanf("%d",&a[i]);
    a[0]=0x7fffffff;
    for (int i=1,l,r;i<=n;i++)
    {
        for (l=0,r=n+1;r-l>1;)
        {
            int mid=(l+r)>>1;
            if (a[dp[mid]]>=a[i]) r=mid;
            else l=mid;
        }
        dp[r]=i,last[i]=dp[r-1];
        if (r>ans) ans=r,idx=i;
    }
    while (idx) dp[tot++]=a[idx],idx=last[idx];
    printf("%d\n",ans);
    for (int i=tot-1;i>=0;i--) printf("%d ",dp[i]);
    puts(""); return 0;
}
\end{lstlisting}

\subsection{硬币问题暴力做法}

\begin{lstlisting}
#include <cstdio>

const int maxn=1000;
double sum,src[maxn+5];

inline int countbit(int num)
{
    int ans=0;
    for (int i=0;i<32;i++) if (num&(1<<i)) ans++;
    return ans;
}

inline void cal(int num)
{
    double p=1;
    for (int i=0;i<32;i++)
        if (num&(1<<i)) p*=src[i];
        else p*=1-src[i];
    sum+=p; return;
}

int main()
{
    int n; scanf("%d",&n);
    for (int i=0;i<n;i++) scanf("%lf",&src[i]);
    for (int i=0;i<1<<n;i++) if (countbit(i)>n/2) cal(i);
    printf("%lf\n",sum);
    return 0;
}
\end{lstlisting}

\subsection{硬币问题递推做法}

\begin{lstlisting}
#include <cstdio>

const int maxn=1000;
double sum,src[maxn+5];
double dp[maxn+5][maxn+5];

int main()
{
    int n; scanf("%d",&n);
    for (int i=1;i<=n;i++) scanf("%lf",&src[i]); dp[0][0]=1;
    for (int i=1;i<=n;i++) dp[i][0]=dp[i-1][0]*(1-src[i]);
    for (int i=1;i<=n;i++)
        for (int j=1;j<=n;j++)
            dp[i][j]=dp[i-1][j-1]*src[i]+dp[i-1][j]*(1-src[i]);
    for (int i=n/2+1;i<=n;i++) sum+=dp[n][i];
    printf("%lf\n",sum);
    return 0;
}
\end{lstlisting}

\subsection{鲜花组合问题普通递推做法}

\begin{lstlisting}
#include <cstdio>
#include <algorithm>
using namespace std;

const int maxn=1000;
int n,m,a[maxn+5];
int dp[maxn+5][maxn+5];

int main()
{
    scanf("%d %d",&n,&m);
    for (int i=1;i<=n;i++) scanf("%d",&a[i]);
    for (int i=0;i<=n;i++) dp[i][0]=1;
    for (int i=1;i<=n;i++)
        for (int j=1;j<=m;j++)
            for (int k=0;k<=min(j,a[i]);k++)
                dp[i][j]+=dp[i-1][j-k];
    printf("%d\n",dp[n][m]);
    return 0;
}
\end{lstlisting}

\subsection{鲜花组合问题优化递推做法}

\begin{lstlisting}
#include <cstdio>
using namespace std;

const int maxn=1000;
int n,m,a[maxn+5];
int dp[maxn+5][maxn+5];

int main()
{
    scanf("%d %d",&n,&m);
    for (int i=1;i<=n;i++) scanf("%d",&a[i]);
    for (int i=0;i<=n;i++) dp[i][0]=1;
    for (int i=1;i<=n;i++)
        for (int j=1;j<=m;j++)
            dp[i][j]=dp[i][j-1]+dp[i-1][j]-((a[i]>=j)?0:dp[i-1][j-1-a[i]]);
    printf("%d\n",dp[n][m]);
    return 0;
}
\end{lstlisting}

\subsection{最大分值问题普通递推做法}

\begin{lstlisting}
#include <cstdio>
#include <algorithm>
using namespace std;

const int maxn=1000;
int n,m,p,a[maxn+5];
int sum[maxn+5],dp[maxn+5][maxn+5];

int main()
{
    scanf("%d %d %d",&n,&m,&p);
    for (int i=1;i<=n;i++) scanf("%d",&a[i]);
    for (int i=1;i<=n;i++) sum[i]=a[i]+sum[i-1],dp[i][0]=0x80000000;
    for (int i=1;i<=n;i++)
        for (int j=1;j<=min(m,i);j++)
            for (int k=0;k<=i-1;k++)
                dp[i][j]=max(dp[i][j],(sum[i]-sum[i-k-1])%p+dp[i-k-1][j-1]);
    printf("%d\n",dp[n][m]);
    return 0;
}
\end{lstlisting}

\subsection{最大分值问题优化递推做法}

\begin{lstlisting}
#include <cstdio>
#include <cstring>
#include <algorithm>
using namespace std;

const int maxn=1000;
int n,m,p,a[maxn+5];
int last[maxn+5][maxn+5];
int sum[maxn+5],dp[maxn+5][maxn+5];

int main()
{
    scanf("%d %d %d",&n,&m,&p);
    for (int i=1;i<=n;i++) scanf("%d",&a[i]);
    for (int i=1;i<=n;i++) sum[i]=a[i]+sum[i-1],dp[i][0]=0x80000000;
    for (int i=1;i<=n;i++)
        for (int j=1;j<=min(m,i);j++)
        {
            for (int k=0;k<p;k++)
                dp[i][j]=max(dp[i][j],(sum[i]-sum[last[j-1][k]])%p+dp[last[j-1][k]][j-1]);
            last[j][dp[i][j]%p]=i;
        }
    printf("%d\n",dp[n][m]);
    return 0;
}
\end{lstlisting}

\end{document}
