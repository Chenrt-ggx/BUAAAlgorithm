\documentclass{article}

% Language setting
% Replace `English' with e.g. `Spanish' to change the document language
\usepackage[english]{babel}
\usepackage[fontset=ubuntu]{ctex}

% Set page size and margins
% Replace `letter paper' with`a4paper' for UK/EU standard size
\usepackage[a4paper,top=1.5cm,bottom=1.5cm,left=2cm,right=2cm,marginparwidth=1.75cm]{geometry}

% Useful packages
\usepackage{amsmath}
\usepackage{graphicx}
\usepackage{listings}
\usepackage[ruled]{algorithm2e}
\usepackage[colorlinks=true, allcolors=blue]{hyperref}

\SetKwProg{Function}{function}{:}{end}

\title{\heiti 计算机学院《算法设计与分析》第一次作业}
\author{114514 李田所}

\begin{document}
\maketitle

\section{计算尽可能紧凑的渐进上界并予以说明}

\subsection{}

$$
T(n)=\left\{
\begin{array}{l}
1\ ,\ n=1\\
T(n-2)+3n\ ,\ n>1
\end{array}
\right.
$$

$T(n)=T(n-2)+3n=T(n-4)+3n+3(n-2)=T(n-6)+3n+3(n-2)+3(n-4)=...$

\qquad\ $=3n+3(n-2)+3(n-4)+...+9+1=\frac{1}{2}(3n+9)(\frac{3n-9}{6}+1)+1=O(n^2)$

由此可知,原式渐进上界为$O(n^2)$。

\subsection{}

$$
T(n)=\left\{
\begin{array}{l}
1\ ,\ n=1\\
T(n/2)+n^2\ ,\ n>1
\end{array}
\right.
$$

使用主方法计算渐进上界,注意到$n^2=O(n^2)$,由上式可知$a=1,b=2,d=2$,由$d>log_b a$,可得上式的渐进上界为$O(n^d)=O(n^2)$。

\subsection{}

$$
T(n)=\left\{
\begin{array}{l}
1\ ,\ n=1\\
T(n/2)+2^n\ ,\ n>1
\end{array}
\right.
$$

断言上式渐进上界为$O(2^n)$,设$T(n)\le{c2^n}$,其中$c$为一足够大常数,下使用替代法进行证明:

\qquad$n=1$时,对$c\ge1$结论显然成立;

\qquad$n>1$时,有:$T(n)=T(n/2)+2^n$

\qquad\qquad\qquad\qquad\qquad\quad$\le{c}2^\frac{n}{2}+2^n$

\qquad\qquad\qquad\qquad\qquad\quad$=c2^n-(c-1)2^n+c2^\frac{n}{2}$

\qquad\qquad\qquad\qquad\qquad\quad$=c2^n-((c-1)2^\frac{n}{2}-c)2^\frac{n}{2}$

\qquad\qquad\qquad\qquad\qquad\quad$\le{c}2^n-(2(c-1)-c)2^\frac{n}{2}$

\qquad\qquad\qquad\qquad\qquad\quad$=c2^n-(c-2)2^\frac{n}{2}$

显然,只要$c\ge2$,就有$T(n)\le{c2^n}$成立,从而原式渐进上界为$O(2^n)$。

\subsection{}

$$
T(n)=\left\{
\begin{array}{l}
1\ ,\ n=1\\
8T(n/4)+2n\ ,\ n>1
\end{array}
\right.
$$

使用主方法计算渐进上界,注意到$2n=O(n)$,由上式可知$a=8,b=4,d=1$,由$d<log_b a$,可得上式的渐进上界为$O(n^{log_b a})=O(n^\frac{3}{2})$。

\subsection{}

$$
T(n)=\left\{
\begin{array}{l}
1\ ,\ n=1\\
T(n/2)+2n log n\ ,\ n>1
\end{array}
\right.
$$

断言上式渐进上界为$O(n log n)$,设$T(n)\le{c n log n}$,其中$c$为一足够大常数,下使用替代法进行证明:

\qquad$n=1$时,对$c\ge1$结论显然成立;

\qquad$n>1$时,有:$T(n)=T(n/2)+2n log n$

\qquad\qquad\qquad\qquad\qquad\quad$\le{c}\frac{n}{2}log(\frac{n}{2})+2n log n$

\qquad\qquad\qquad\qquad\qquad\quad$\le{c}\frac{n}{2}log(n)+2n log n$

\qquad\qquad\qquad\qquad\qquad\quad$=c n log n-(\frac{c}{2}-2)n log n$

显然,只要$c\ge4$,就有$T(n)\le{c n log n}$成立,从而原式渐进上界为$O(n log n)$。

\subsection{}

$$
T(n)=\left\{
\begin{array}{l}
1\ ,\ n=1\\
3T(n/2)+n\ ,\ n>1
\end{array}
\right.
$$

使用主方法计算渐进上界,注意到$n=O(n)$,由上式可知$a=3,b=2,d=1$,由$d<log_b a$,可得上式的渐进上界为$O(n^{log_b a})=O(n^{log_2 3})$。

\subsection{}

$$
T(n)=\left\{
\begin{array}{l}
1\ ,\ n=1\\
T(n/2)+n/2\ ,\ n>1
\end{array}
\right.
$$

使用主方法计算渐进上界,注意到$n/2=O(n)$,由上式可知$a=1,b=2,d=1$,由$d>log_b a$,可得上式的渐进上界为$O(n^d)=O(n)$。

\section{k路归并问题}

现有$k$个有序数组(从小到大排序),每个数组中包含$n$个元素。你的任务是将他们合并成$1$个包含$kn$个元素的有序数组。首先来回忆一下课上讲的归并排序算法,它提供了一种合并有序数组的算法$Merge$。如果我们有两个有序数组的大小分别为$x$和$y$,$Merge$算法可以用$O(x+y)$的时间来合并这两个数组。

\subsection{}

如果我们应用$Merge$算法先合并第一个和第二个数组,然后由合并后的数组与第三个合并,再与第四个合并,直到合并完$k$个数组。请分析这种合并策略的时间复杂度(请用关于$k$和$n$的函数表示)。

显然,第一次合并的复杂度为$O(n+n)$,第二次合并的复杂度为$O(2n+n)$,第三次合并的复杂度为$O(3n+n)$,以此类推,第$k-1$次合并的复杂度为$O((k-1)n+n)$。

从而总复杂度为$n+2n+...+(k-1)n=\frac{1}{2}k(k-1)n=O(nk^2)$。

\subsection{}

针对本题的任务,请给出一个更高效的算法,并分析它的时间复杂度(此题若取得满分,所设计算法的时间复杂度应为$O(n k log k)$)。

分析题目,只要使用优先队列维护$k$个有序数组未归并部分首个元素的集合即可,每次从优先队列中取出一个元素归并,再将提供被取出元素的数组的下一元素插入优先队列,这样共需要处理$k n$个元素,又因为优先队列的大小为$k$,每个元素的处理时间为$log k$,即得到$O(n k log k)$的时间复杂度,伪代码如下。

\begin{algorithm}[H]

\caption{k路归并问题优化做法}
\LinesNumbered
\KwIn{正整数$n$,为数组长度,正整数$k$,为数组个数,$k$个长度为$n$的数组$A[]$}
\KwOut{长度为$n k$的数组$Ans$,为归并后的数组}

$global\ pointer[k]$

\Function{$main(n,k,A)$}{
    \For{$i\leftarrow0\ to\ k-1$}{
        $pointer[i] \leftarrow 0$\;
        $push\ (A[i][pointer[i]],i)\ to\ priority\_queue$\;
    }
    \For{$i\leftarrow0\ to\ n k-1$}{
        $(val,index) \leftarrow top\ of\ priority\_queue$\;
        $remove\ top\ from\ priority\_queue$\;
        $Ans[i] \leftarrow val$\;
        $pointer[index] \leftarrow pointer[index]+1$\;
        \If{$pointer[index]<{length\ of\ A[index]}$}{
            $push\ (A[index][pointer[index]],index)\ to\ priority\_queue$\;
        }
    }
    \textbf{$return$}\;
}

\end{algorithm}

已知优先队列的插入复杂度$O(log size)$,删除复杂度$O(log size)$,查询队首复杂度$O(1)$,注意到优先队列的$size$不超过$k$,第$3-6$行循环中,第$4$行复杂度为$O(1)$,第$5$行复杂度为$O(log k)$,单次循环复杂度为$O(log k)$,总复杂度为$O(k log k)$。

第$7-15$行循环中,第$9$和$13$行复杂度为$O(log k)$,其余行复杂度为$O(1)$,单次循环复杂度同样为为$O(log k)$,总复杂度为$O(n k log k)$,可知$k$路归并问题优化做法的总时间复杂度为$O(n k log k)$。

\section{填数字问题}

给定一个长度为$n$的数组$A[1..n]$,初始时数组中所有元素的值均为$0$,现对其进行$n$次操作。第$i$次操作可分为两个步骤:

\begin{enumerate}
    \item 先选出$A$数组长度最长且连续为$0$的区间,如果有多个这样的区间,则选择最左端的区间,记本次选定的闭区间为$[l,r]$;
    \item 对于闭区间$[l,r]$,将$A[\lfloor\frac{l+r}{2}\rfloor]$赋值为$i$,其中$\lfloor{x}\rfloor$表示对数$x$做向下取整。
\end{enumerate}

例如$n=6$的情形,初始时数组为$A=[0,0,0,0,0,0]$。

第一次操作为选择区间$[1,6]$,赋值后为$A=[0,0,1,0,0,0]$;

第二次操作为选择区间$[4,6]$,赋值后为$A=[0,0,1,0,2,0]$;

第三次操作为选择区间$[1,2]$,赋值后为$A=[3,0,1,0,2,0]$;

第四次操作为选择区间$[2,2]$,赋值后为$A=[3,4,1,0,2,0]$;

第五次操作为选择区间$[4,4]$,赋值后为$A=[3,4,1,5,2,0]$;

第六次操作为选择区间$[6,6]$,赋值后为$A=[3,4,1,5,2,6]$,为所求。

请设计一个高效的算法求出$n$次操作后的数组,并分析其时间复杂度。

\subsection{算法思想}

一个朴素的想法是按照题目要求模拟,即维护一个连续$0$区间的集合,每次选择最左最长连续$0$区间处理,处理完成后移除这个区间,并加入新增的至多两个区间。

显然连续$0$区间的集合最大长度为$n$,在连续$0$区间的集合中选择最左最长连续$0$区间时,朴素做法为$O(n)$复杂度,使用优先队列或平衡树维护可以做到$O(log n)$的复杂度,此时总复杂度$O(n log n)$。

考虑更高效的做法,我们注意到长度长的区间总在长度短的区间前被处理,对相同长度的区间,处理顺序是从左向右,因此只要自顶向下搜索全部区间,每次搜索时先递归搜索左边再递归搜索右边,用邻接表维护每个区间长度被搜索到的序列即可。

又因为普通的邻接表新节点插入位置是链表头部,遍历时后插入的节点先被访问,不符合我们的要求,我们需要进行一些调整,一种方式是维护链表尾指针,将新节点插入位置调整为链表尾部,更简单的方式是搜索区间时先递归搜索右边,再递归搜索左边,使同一长度的区间从右向左被插入链表,遍历时顺序即为从左向右的顺序。

最后我们只要按从大到小的顺序遍历邻接表即可,由于搜索区间时$O(n)$复杂度遍历了全部$n$个区间,遍历邻接表时$O(n)$复杂度遍历了全部$n$个区间,总复杂度$O(n)$。

上述主要做法均使用$C++$进行了实现并进行了测试,为使格式美观相关代码见文末附录。

\subsection{朴素做法伪代码与复杂度分析}

\begin{algorithm}[H]

\caption{填数字问题朴素做法}
\LinesNumbered
\KwIn{正整数$n$,为数组长度}
\KwOut{长度为$n$的数组$A$,为填充后的数组}

\Function{$main(n)$}{
    $push\ (0,n-1)\ to\ priority\_queue$\;
    \For{$i\leftarrow1\ to\ n$}{
        $(l,r) \leftarrow top\ of\ priority\_queue$\;
        $remove\ top\ from\ priority\_queue$\;
        $mid \leftarrow (l+r)/2$\;
        $A[mid] \leftarrow i$\;
        \If{$mid-1\ge{l}$}{
            $push\ (l,mid-1)\ to\ priority\_queue$\;
        }
        \If{$r\ge{mid+1}$}{
            $push\ (mid+1,r)\ to\ priority\_queue$\;
        }
    }
    \textbf{$return\ A$}\;
}

\end{algorithm}

已知优先队列的插入复杂度$O(log size)$,删除复杂度$O(log size)$,查询队首复杂度$O(1)$,注意到区间与中点的一一对应性,可知优先队列的$size$不超过$n$,因此第$4$行复杂度为$O(1)$,第$5$行复杂度为$O(log n)$,第$9$和$12$行复杂度为$O(log n)$,第$4-13$行即循环体内的总复杂度为$O(log n)$。

注意到在第$3-14$行的循环中,复杂度为$O(log n)$执行了$n$次,而第$2$行复杂度为$O(1)$,可知填数字问题朴素做法的总时间复杂度为$O(n log n)$。

\subsection{优化做法伪代码与复杂度分析}

\begin{algorithm}[H]

\caption{填数字问题优化做法}
\LinesNumbered
\KwIn{正整数$n$,为数组长度}
\KwOut{长度为$n$的数组$A$,为填充后的数组}

$global\ list[n]$\;

\Function{$search(l,r)$}{
    $mid \leftarrow (l+r)/2$\;
    $insert\ mid\ to\ head\ of\ list[r-l+1]$\;
    \If{$r\ge{mid+1}$}{
        $call\ search(mid+1,r)$\;
    }
    \If{$mid-1\ge{l}$}{
        $call\ search(l,mid-1)$\;
    }
}

\Function{$main(n)$}{
    $call\ search(0,n-1)$\;
    $count \leftarrow 1$\;
    \For{$i\leftarrow{n}\ to\ 1$}{
        \For{$j\leftarrow{head\ of\ list[i]}\ to\ tail\ of\ list[i]$}{
            $A[value\ of\ j] \leftarrow count$\;
            $count \leftarrow count+1$\;
        }
    }
    \textbf{$return\ A$}\;
}

\end{algorithm}

已知链表的头部插入复杂度$O(1)$,可知$search$函数中非递归调用的部分是$O(1)$的,且满足:

$$
T(n)=\left\{
\begin{array}{l}
1\ ,\ n=1\\
T(n/2)+T(n/2)+O(1)\ ,\ n>1
\end{array}
\right.
$$

显然可以使用主方法计算上式的复杂度,由上式可知$a=2,b=2,d=0$,由$d<log_b a$,可得$search$函数总复杂度为$O(n^{log_b a})=O(n)$。

显然$17-18$行的复杂度是$O(1)$的,而$16-18$行的循环所做的工作是对$A$数组赋值,因此在$15-19$行的循环中总执行次数是$n$,也就是说$15-19$行的循环的复杂度是$O(n)$的,又因为$search$函数总复杂度亦为$O(n)$,可知填数字问题优化做法的总时间复杂度为$O(n)$。

\section{奇数因子和问题}

定义$god(i)$为整数$i$的最大奇数因子,例如$god(3)=3$,$god(14)=7$。

请你设计一个高效算法,计算一个整数区间$[A,B]$内所有数的最大奇数因子和,即$\sum_{i\in{A},B]}god(i)$,并分析该算法的时间复杂度。

例如,区间$[3,9]$的计算结果为$3+1+5+3+7+1+9=29$。

\subsection{算法思想}

一个朴素的想法是按照题目要求模拟,即遍历区间$[A,B]$,处理每一个数的结果并累加这些结果。

对整数$i$,显然$god(i)$的计算方式是对$i$一直除$2$,直到无法整除为止,这个操作可以通过一个循环完成,但这样不够优雅,注意到对$i$一直除$2$,直到无法整除为止实质上是消除$i$二进制意义下末尾全部的$0$,而采用位运算$i\&(-i)$可以仅保留$i$的最低位$1$而将其它为置$0$,因此,对$0$有$god(i)=0$,对非$0$整数$i$有$god(i)=i/(i\&(-i))$。

朴素的处理下,我们要遍历区间长度个数,对每个数可以在$O(1)$时间内计算出结果,总复杂度$O(n)$,这里$n$的意义是区间长度。

考虑更高效的做法,我们注意到$\sum_{i\in{A},B]}$具有对称性,即$\sum_{i\in{-A},-B]}god(i)=-\sum_{i\in{A},B]}god(i)$且$\sum_{i\in{-A},A]}god(i)=0$,整数区间$[A,B]$上的问题即变为正整数区间$[A,B]$上的问题,我们注意到$\sum_{i\in{A},B]}god(i)=\sum_{i\in{1},B]}god(i)-\sum_{i\in{1},A-1]}god(i)$,正整数区间$[A,B]$上的问题即变为正整数区间$[1,n]$上的问题,下面只要考虑这个问题。

在正整数区间$[1,n]$上,先对所有数求和,再减去全部偶数的和的一半,再减去全部四的倍数的和的四分之一,再减去全部八的倍数的和的八分之一,以此类推直到减去全部$2^{\lfloor{log_2n}\rfloor}$的倍数的和的$2^{\lfloor{log_2n}\rfloor}$分之一,总共进行了$O(log n)$次减法,此时即得答案。

在这种处理下,使用等差数列求和可以在$O(1)$时间内计算出每次减去的数,总复杂度$O(n log n)$,这里$n$的意义是$max(|A|,|B|)$。

上述主要做法均使用$C++$进行了实现并进行了测试,为使格式美观相关代码见文末附录。

\subsection{朴素做法伪代码与复杂度分析}

\begin{algorithm}[H]

\caption{奇数因子和问题朴素做法}
\LinesNumbered
\KwIn{整数$A,B$,为区间端点}
\KwOut{整数$Ans$,为区间奇数因子和}

\Function{$main(A,B)$}{
    $Ans \leftarrow 0$\;
    \For{$i\leftarrow{A}\ to\ B$}{
        \If{$i\ne0$}{
            $Ans \leftarrow Ans+i/(i\&-i)$\;
        }
    }
    \textbf{$return\ Ans$}\;
}

\end{algorithm}

显然第$4-6$行的复杂度是$O(1)$的,$3-7$行的总循环次数为$B-A+1$,可知奇数因子和问题朴素做法的总时间复杂度为$O(n)$,这里$n$的意义是区间长度即$length=B-A$。

\subsection{优化做法伪代码与复杂度分析}

\begin{algorithm}[H]

\caption{奇数因子和问题优化做法}
\LinesNumbered
\KwIn{整数$A,B$,为区间端点}
\KwOut{整数$Ans$,为区间奇数因子和}

\Function{$calculate(val)$}{
    \If{$val=0$}{
        \textbf{$return\ 0$}\;
    }
    $answer \leftarrow (val+1)*val/2$\;
    $top \leftarrow \lfloor{log_2val}\rfloor$\;
    \For{$i\leftarrow1\ to\ top$}{
        $answer \leftarrow answer-(((val>>i)+1)*(val>>i)/2)$\;
    }
    \textbf{$return\ answer$}\;
}

\Function{$main(A,B)$}{
    \If{$A>B$}{
        \textbf{$return\ 0$}\;
    }
    \If{$A<0\ and\ B<0$}{
        \textbf{$return\ (call\ calculate(-r-1))-(call\ calculate(-l))$}\;
    }
    \If{$A\ge0\ and\ B\ge0$}{
        \textbf{$return\ (call\ calculate(r))-(call\ calculate(l-1))$}\;
    }
    \If{$A<0\ and\ B\ge0$}{
        \textbf{$return\ (call\ calculate(r))-(call\ calculate(-l))$}\;
    }
    $//\ this\ will\ not\ happen$

    \textbf{$return\ -1$}\;
}

\end{algorithm}

对$calculate$函数,显然第$2-6$行和第$8$行的复杂度都是$O(1)$的,而第$7-9$行的循环的执行次数为$O(log n)$,因此$calculate$函数的总复杂度数$O(log n)$的。

注意到$calculate$函数会且仅会被调用至多两次,传入的参数不超过$max(|A|+1,|B|+1)$,可知奇数因子和问题优化做法的总时间复杂度为$O(log n)$,这里$n$的意义是$A,B$中绝对值较大者即$max(|A|,|B|)$。

\section{区间计数问题}

给定一个整数数组$C=[c_1,c_2,···,c_n]$,询问有多少个区间的区间和小于等于常数$X$,即询
问有多少对$l,r$满足$l\le{r}$且$\sum_{i=1}c_i\le{x}$。

例如,给定常数$X=3$,数组$C=[1,2,3]$,$l=1,r=1$,$l=2,r=2$,$l=3,r=3$,和
$l=1,r=2$均满足条件,因此答案为$4$。

请设计一个高效的算法来解决此问题,并分析该算法的时间复杂度。

\subsection{算法思想}

一个朴素的想法是枚举全部可能的区间,对每个区间求和并判断是否满足小于等于常数$X$,枚举复杂度$O(n^2)$区间求和的朴素做法为$O(n)$复杂度,使用前缀和优化后为$O(1)$复杂度,此时总复杂度$O(n^2)$。

考虑对上述方法的优化,记前缀和数组为$sum[]$,在我们枚举区间时,要做的实质上是对每个$i$,寻找在$i$的右侧有多少个$j$,满足$sum[j]-sum[i]<=X$,可以使用分治算法完成这一过程,也可以使用平衡树或字典树直接维护区间排名,下面分别进行说明。

对于分治做法,我们将前缀和数组从中间分开成两个子问题递归求解,通过在每次处理结束时进行一次归并排序的归并过程,保证前缀和数组左右两侧分别有序,此时维护指向左子数组最右端的指针$A$和指向右子数组最右端的指针$B$,先向左移动$B$使$sum[B]-sum[A]<=X$,使结果增加$B-mid$,再向左移动$A$一个单位,继续上述过程,由于$A$和$B$分别访问左子数组和右子数组每个元素一次复杂度为$O(n)$,总复杂度$O(n log n)$。

对于用平衡树或字典树的做法,从右向左枚举,每枚举到一个下标$i$,先查询$sum[i]+X+1$在平衡树或字典树中的排名,这里$x$排名定义为比$x$小的数的个数加一,再将$sum[i]$插入平衡树或字典树即可,单次查询复杂度为$O(log n)$或$O(log k)$,总复杂度为$O(n log n)$或$O(n log k)$,这里$k$的意义是$max(c_1,c_2,···,c_n)-min(c_1,c_2,···,c_n)$。

上述主要做法均使用$C++$进行了实现并进行了测试,为使格式美观相关代码见文末附录。

\subsection{分治做法伪代码与复杂度分析}

\begin{algorithm}[H]

\caption{区间计数问题分治做法}
\LinesNumbered
\KwIn{长度为$n$的原始数组$C$,常数$X$}
\KwOut{整数$Ans$,为符合要求的区间个数}

$global\ Ans$\;
$global\ sum[n]$\;

\Function{$solve(l,r)$}{
    \If{$l=r$}{
        \textbf{$return$}\;
    }
    $mid \leftarrow (l+r)/2$\;
    $call\ solve(l,mid)$\;
    $call\ solve(mid+1,r)$\;
    $r\_pointer \leftarrow r$\;
    \For{$i\leftarrow{mid}\ to\ l$}{
        \While{$r\_pointer>mid\ and\ sum[r\_pointer]-sum[i]>X$}{
            $r\_pointer \leftarrow r\_pointer-1$\;
        }
        \If{$r\_pointer=mid$}{
            $break$\;
        }
        $Ans \leftarrow Ans+r\_pointer-mid$\;
    }
    $//\ this\ is\ as\ same\ as\ merge\ sort$

    $merge\ sorted\ left\ sub\ array\ and\ sorted\ right\ sub\ array\ to\ one\ sorted\ array$\;
}

\Function{$main(C,X)$}{
    $Ans \leftarrow 0$\;
    $sum[0] \leftarrow C[0]$\;
    \For{$i\leftarrow1\ to\ n$}{
        $sum[i] \leftarrow sum[i-1]+C[i]$\;
    }
    \For{$i\leftarrow0\ to\ n$}{
        \If{$sum[i]\le{X}$}{
            $Ans \leftarrow Ans+1$\;
        }
    }
    $call\ solve(0,n-1)$\;
    \textbf{$return\ Ans$}\;
}

\end{algorithm}

考虑$solve$函数中非递归调用的部分,第$4-7$行的复杂度是$O(1)$的,对$11-19$行的循环,注意到$i$从大到小单向遍历了整个左子数组,$r\_pointer$从大到小单向遍历了整个右子数组,即在此循环中前缀和数组在$l$到$r$的部分被遍历了一次,因此这部分的复杂度是$O(n)$的,熟知归并排序的归并过程的复杂度也是$O(n)$的,我们有:

$$
T(n)=\left\{
\begin{array}{l}
1\ ,\ n=1\\
T(n/2)+T(n/2)+O(n)\ ,\ n>1
\end{array}
\right.
$$

显然可以使用主方法计算上式的复杂度,由上式可知$a=2,b=2,d=1$,由$d=log_b a$,可得$solve$函数总复杂度为$O(n^d log n)=O(n log n)$。

显然$26-28$行的循环计算前缀和复杂度是$O(n)$的,$29-33$行的循环特判复杂度是$O(n)$的,可知区间计数问题分治做法的总时间复杂度为$O(n log n)$。

\subsection{平衡树或字典树做法伪代码与复杂度分析}

\begin{algorithm}[H]

\caption{区间计数问题平衡树或字典树做法}
\LinesNumbered
\KwIn{长度为$n$的原始数组$C$,常数$X$}
\KwOut{整数$Ans$,为符合要求的区间个数}

$global\ sum[n]$\;

\Function{$main(C,X)$}{
    $Ans \leftarrow 0$\;
    $sum[0] \leftarrow C[0]$\;
    \For{$i\leftarrow1\ to\ n$}{
        $sum[i] \leftarrow sum[i-1]+C[i]$\;
    }
    \For{$i\leftarrow0\ to\ n$}{
        \If{$sum[i]\le{X}$}{
            $Ans \leftarrow Ans+1$\;
        }
    }
    $insert\ sum[n-1]\ to\ tree$\;
    \For{$i\leftarrow{n-2}\ to\ 0$}{
        $ans \leftarrow ans+(query\ rank\ of\ sum[i]+X+1\ in\ tree)$\;
        $insert\ sum[i]\ to\ tree$\;
    }
    \textbf{$return\ Ans$}\;
}

\end{algorithm}

已知任意一种平衡树的插入复杂度$O(log n)$,查询排名复杂度$O(log n)$,这里$n$的意义是元素个数,$01$字典树的插入复杂度$O(log n)$,查询排名复杂度$O(log n)$,这里$n$的意义是元素极差。

显然$5-7$行的循环计算前缀和复杂度是$O(n)$的,$8-12$行的循环特判复杂度是$O(n)$的,对$14-17$行的循环,每次循环的复杂度为一次查询排名和一次插入的复杂度,平衡树和字典树的复杂度分别为$O(log n)$和$O(log k)$,区间计数问题平衡树或字典树做法的总时间复杂度为$O(n log n)$或$O(n log k)$,这里$k$的意义是$max(c_1,c_2,···,c_n)-min(c_1,c_2,···,c_n)$。

\section{附录暨部分算法的实现}

\subsection{填数字问题朴素做法与测试数据生成}

\definecolor{mygreen}{rgb}{0,0.6,0}
\definecolor{mygray}{rgb}{0.5,0.5,0.5}
\definecolor{mymauve}{rgb}{0.58,0,0.82}

\lstset{
    backgroundcolor=\color{white},   % choose the background color; you must add \usepackage{color} or \usepackage{xcolor}
    basicstyle=\footnotesize,        % the size of the fonts that are used for the code
    breakatwhitespace=false,         % sets if automatic breaks should only happen at whitespace
    breaklines=true,                 % sets automatic line breaking
    captionpos=bl,                   % sets the caption-position to bottom
    commentstyle=\color{mygreen},    % comment style
    deletekeywords={...},            % if you want to delete keywords from the given language
    escapeinside={\%*}{*)},          % if you want to add LaTeX within your code
    extendedchars=true,              % lets you use non-ASCII characters; for 8-bits encodings only, does not work with UTF-8
    frame=single,                    % adds a frame around the code
    keepspaces=true,                 % keeps spaces in text, useful for keeping indentation of code (possibly needs columns=flexible)
    keywordstyle=\color{blue},       % keyword style
    language=c++,                    % the language of the code
    morekeywords={*,...},            % if you want to add more keywords to the set
    numbers=left,                    % where to put the line-numbers; possible values are (none, left, right)
    numbersep=5pt,                   % how far the line-numbers are from the code
    numberstyle=\tiny\color{mygray}, % the style that is used for the line-numbers
    rulecolor=\color{black},         % if not set, the frame-color may be changed on line-breaks within not-black text (e.g. comments (green here))
    showspaces=false,                % show spaces everywhere adding particular underscores; it overrides 'showstringspaces'
    showstringspaces=false,          % underline spaces within strings only
    showtabs=false,                  % show tabs within strings adding particular underscores
    stepnumber=1,                    % the step between two line-numbers. If it's 1, each line will be numbered
    stringstyle=\color{mymauve},     % string literal style
    tabsize=2,                       % sets default tabsize to 2 spaces
}

\begin{lstlisting}
#include <queue>
#include <cstdio>
using namespace std;

struct Node
{
    int l, r, len;

    inline Node(int l, int r)
    {
        this->l = l, this->r = r;
        len = r - l + 1;
        return;
    }

    inline bool operator<(const Node &rhs) const
    {
        if (this->len < rhs.len)
            return true;
        if (this->len > rhs.len)
            return false;
        return this->l > rhs.l;
    }

    inline bool operator>(const Node &rhs) const
    {
        if (this->len > rhs.len)
            return true;
        if (this->len < rhs.len)
            return false;
        return this->l < rhs.l;
    }
};

int a[10005];

int main()
{
    int n = 10000;
    freopen("testcase.txt", "w", stdout);
    printf("%d\n", n);
    priority_queue<Node> pq;
    Node node(0, n - 1);
    pq.push(node);
    for (int i = 1, l, r; i <= n; i++)
    {
        Node node = pq.top();
        pq.pop();
        l = node.l, r = node.r;
        int mid = (l + r) >> 1;
        a[mid] = i;
        if (mid - 1 >= l)
        {
            Node newNode(l, mid - 1);
            pq.push(newNode);
        }
        if (r >= mid + 1)
        {
            Node newNode(mid + 1, r);
            pq.push(newNode);
        }
    }
    for (int i = 0; i < n; i++)
        printf("%d\n", a[i]);
    return 0;
}
\end{lstlisting}

\subsection{填数字问题优化做法与自动对拍}

\begin{lstlisting}
#include <cstdio>
#include <vector>
using namespace std;

int a[10005];
vector<int> map[10005];

void build(int l, int r)
{
    int mid = (l + r) >> 1;
    map[r - l + 1].push_back(mid);
    if (mid - 1 >= l)
        build(l, mid - 1);
    if (r >= mid + 1)
        build(mid + 1, r);
    return;
}

int main()
{
    int n, cnt = 0;
    freopen("testcase.txt", "r", stdin);
    scanf("%d", &n);
    build(0, n - 1);
    for (int i = n; i > 0; i--)
        for (int j = 0; j < (int)map[i].size(); j++)
            a[map[i][j]] = ++cnt;
    for (int i = 0; i < n; i++)
        printf("%d\n", a[i]);
    bool isRight = true;
    for (int i = 0, std; i < n; i++)
    {
        scanf("%d", &std);
        if (a[i] != std)
            isRight = false;
    }
    printf(isRight ? "AC\n" : "WA\n");
    return 0;
}
\end{lstlisting}

\subsection{奇数因子和问题朴素做法与测试数据生成}

\begin{lstlisting}
#include <cstdio>
#include <random>
#include <algorithm>
using namespace std;

mt19937 generator(0x19260817);

int main()
{
    int n = 10000;
    freopen("testcase.txt", "w", stdout);
    int l = generator() % n - (n >> 1), r = generator() % n - (n >> 1);
    if (l > r)
        swap(l, r);
    printf("%d %d\n", l, r);
    int ans = 0;
    for (int i = l; i <= r; i++)
        ans += i ? i / (i & -i) : 0;
    printf("%d\n", ans);
    return 0;
}
\end{lstlisting}

\subsection{奇数因子和问题优化做法与自动对拍}

\begin{lstlisting}
#include <cmath>
#include <cstdio>

inline int cal(int num)
{
    if (num == 0)
        return 0;
    int ans = (num + 1) * num >> 1;
    int top = int(log(num) / log(2));
    for (int i = 1; i <= top; i++)
        ans -= ((num >> i) + 1) * (num >> i) >> 1;
    return ans;
}

int main()
{
    int l, r, ans = 0;
    freopen("testcase.txt", "r", stdin);
    scanf("%d %d", &l, &r);
    if (l > r)
        ans = 0;
    else if (l < 0 && r < 0)
        ans = cal(-r - 1) - cal(-l);
    else if (l >= 0 && r >= 0)
        ans = cal(r) - cal(l - 1);
    else if (l < 0 && r >= 0)
        ans = cal(r) - cal(-l);
    printf("%d\n", ans);
    int std;
    scanf("%d", &std);
    printf(ans == std ? "AC\n" : "WA\n");
    return 0;
}
\end{lstlisting}

\subsection{区间计数问题暴力做法与测试数据生成}

\begin{lstlisting}
#include <cstdio>
#include <random>
using namespace std;

int a[10005];
int sum[10005];
mt19937 generator(0x19260817);

inline int cal(int l, int r)
{
    if (l == 0)
        return sum[r];
    return sum[r] - sum[l - 1];
}

int main()
{
    freopen("testcase.txt", "w", stdout);
    int n = 10000, tot = 0;
    int spl = generator() % 10 - 5;
    for (int i = 0; i < n; i++)
        sum[i] = a[i] = generator() % n - (n >> 1);
    for (int i = 1; i < n; i++)
        sum[i] = a[i] + sum[i - 1];
    for (int i = 0; i < n; i++)
        for (int j = i; j < n; j++)
            tot += cal(i, j) <= spl;
    printf("%d %d\n", n, spl);
    for (int i = 0; i < n; i++)
        printf("%d\n", a[i]);
    printf("%d\n", tot);
    return 0;
}
\end{lstlisting}

\subsection{区间计数问题分治做法与自动对拍}

\begin{lstlisting}
#include <cstdio>

int ans, spl, a[10005], sum[10005], buf[10005];

inline void merge(int l, int mid, int r)
{
    for (int i = l; i <= r; i++)
        buf[i] = sum[i];
    int i = l, j = mid + 1, k = l;
    while (i <= mid || j <= r)
    {
        if (i > mid)
            sum[k++] = buf[j++];
        else if (j > r)
            sum[k++] = buf[i++];
        else if (buf[i] > buf[j])
            sum[k++] = buf[j++];
        else
            sum[k++] = buf[i++];
    }
    return;
}

void solve(int l, int r)
{
    if (l == r)
        return;
    int mid = (l + r) >> 1;
    solve(l, mid), solve(mid + 1, r);
    for (int lr = mid, rr = r; lr >= l; lr--)
    {
        while (sum[rr] - sum[lr] > spl && rr > mid)
            rr--;
        if (rr == mid)
            break;
        ans += rr - mid;
    }
    merge(l, mid, r);
    return;
}

int main()
{
    int n;
    freopen("testcase.txt", "r", stdin);
    scanf("%d %d", &n, &spl);
    for (int i = 0; i < n; i++)
        scanf("%d", a + i), sum[i] = a[i];
    for (int i = 1; i < n; i++)
        sum[i] = sum[i - 1] + a[i];
    for (int i = 0; i < n; i++)
        ans += sum[i] <= spl;
    solve(0, n - 1);
    printf("%d\n", ans);
    int std;
    scanf("%d", &std);
    printf(ans == std ? "AC\n" : "WA\n");
    return 0;
}
\end{lstlisting}

\subsection{区间计数问题字典树做法与自动对拍}

\begin{lstlisting}
#include <cstdio>
#include <cstdlib>

const int bit = 25;
const int maxn = 10005;
const int inf = 100000000;

int cnt = 2;
int size[maxn * bit];
int son[maxn * bit][2];
int ans, spl, a[maxn], sum[maxn];

inline int newnode()
{
    son[cnt][0] = son[cnt][1] = size[cnt] = 0;
    return cnt++;
}

inline void ins(int num)
{
    for (int i = bit - 1, cur = 1; i >= 0; i--)
    {
        int which = (num >> i) & 1;
        if (!son[cur][which])
            son[cur][which] = newnode();
        cur = son[cur][which], size[cur]++;
    }
    return;
}

inline int rank(int num)
{
    int ans = 0;
    for (int i = bit - 1, cur = 1; i >= 0; i--)
    {
        if ((num >> i) & 1)
            ans += size[son[cur][0]];
        cur = son[cur][(num >> i) & 1];
    }
    return ans;
}

int main()
{
    int n;
    freopen("testcase.txt", "r", stdin);
    scanf("%d %d", &n, &spl);
    for (int i = 0; i < n; i++)
        scanf("%d", a + i), sum[i] = a[i];
    for (int i = 1; i < n; i++)
        sum[i] = sum[i - 1] + a[i];
    ins(sum[n - 1] + inf);
    for (int i = n - 2; i >= 0; i--)
    {
        ans += rank(sum[i] + spl + 1 + inf);
        ins(sum[i] + inf);
    }
    for (int i = 0; i < n; i++)
        ans += sum[i] <= spl;
    printf("%d\n", ans);
    int std;
    scanf("%d", &std);
    printf(ans == std ? "AC\n" : "WA\n");
    return 0;
}
\end{lstlisting}

\subsection{区间计数问题平衡树做法与自动对拍}

\begin{lstlisting}
#include <cstdio>
#include <cstdlib>

int ans, spl, a[10005], sum[10005];

struct tree
{
    struct tree *ls;
    struct tree *rs;
    int siz, con, data, height;
};

typedef struct tree stu;
typedef struct tree *ptr;

int size(ptr now)
{
    if (now == NULL)
        return 0;
    return now->siz;
}

int h(ptr now)
{
    if (now == NULL)
        return 0;
    return now->height;
}

void pushup(ptr now)
{
    if (now == NULL)
        return;
    now->height = 1;
    now->siz = now->con;
    now->siz += size(now->ls);
    now->siz += size(now->rs);
    if (h(now->ls) > h(now->rs))
        now->height += h(now->ls);
    else
        now->height += h(now->rs);
    return;
}

void left(ptr *now)
{
    ptr tmp = (*now)->rs;
    (*now)->rs = tmp->ls;
    tmp->ls = *now;
    tmp->siz = (*now)->siz;
    pushup(*now), pushup(tmp);
    *now = tmp;
    return;
}

void right(ptr *now)
{
    ptr tmp = (*now)->ls;
    (*now)->ls = tmp->rs;
    tmp->rs = *now;
    tmp->siz = (*now)->siz;
    pushup(*now), pushup(tmp);
    *now = tmp;
    return;
}

void balance(ptr *now)
{
    if (*now == NULL)
        return;
    if (h((*now)->ls) - h((*now)->rs) == 2)
    {
        if (h((*now)->ls->ls) > h((*now)->ls->rs))
            right(now);
        else
            left(&(*now)->ls), right(now);
        return;
    }
    if (h((*now)->rs) - h((*now)->ls) == 2)
    {
        if (h((*now)->rs->rs) > h((*now)->rs->ls))
            left(now);
        else
            right(&(*now)->rs), left(now);
        return;
    }
    return;
}

void ins(ptr *now, int num)
{
    if (*now == NULL)
    {
        *now = (ptr)malloc(sizeof(stu));
        (*now)->siz = (*now)->con = 1;
        (*now)->data = num, (*now)->height = 0;
        (*now)->ls = (*now)->rs = NULL;
        return;
    }
    if ((*now)->data == num)
    {
        (*now)->con++;
        pushup(*now);
        return;
    }
    if ((*now)->data > num)
        ins(&(*now)->ls, num);
    else
        ins(&(*now)->rs, num);
    pushup(*now);
    balance(now);
    return;
}

int rank(ptr now, int num)
{
    if (now == NULL)
        return 0;
    if (now->data == num)
        return size(now->ls);
    if (now->data > num)
        return rank(now->ls, num);
    return size(now->ls) + now->con + rank(now->rs, num);
}

int main()
{
    int n;
    freopen("testcase.txt", "r", stdin);
    scanf("%d %d", &n, &spl);
    for (int i = 0; i < n; i++)
        scanf("%d", a + i), sum[i] = a[i];
    for (int i = 1; i < n; i++)
        sum[i] = sum[i - 1] + a[i];
    ptr root = NULL;
    ins(&root, sum[n - 1]);
    for (int i = n - 2; i >= 0; i--)
    {
        ans += rank(root, sum[i] + spl + 1);
        ins(&root, sum[i]);
    }
    for (int i = 0; i < n; i++)
        ans += sum[i] <= spl;
    printf("%d\n", ans);
    int std;
    scanf("%d", &std);
    printf(ans == std ? "AC\n" : "WA\n");
    return 0;
}
\end{lstlisting}

\end{document}
